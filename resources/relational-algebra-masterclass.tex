\documentclass[10pt]{article}

\usepackage{verbatim, multicol, tabularx, graphicx, float, xcolor, colortbl}
\usepackage{amsmath,amsthm, amssymb, latexsym, listings, qtree}

\lstset{frame=tb,
  language=Java,
  aboveskip=1mm,
  belowskip=0mm,
  showstringspaces=false,
  columns=flexible,
  basicstyle={\ttfamily},
  numbers=none,
  frame=single,
  breaklines=true,
  breakatwhitespace=true
}

\textwidth = 6.5 in
\textheight = 9 in
\oddsidemargin = 0.0 in
\evensidemargin = 0.0 in
\topmargin = -0.25 in
\headheight = 0.0 in
\headsep = 0.0 in
\parskip = 0.0in
\parindent = 0.0in

\def\ojoin{\setbox0=\hbox{$\bowtie$}%
  \rule[-.02ex]{.25em}{.4pt}\llap{\rule[\ht0]{.25em}{.4pt}}}
\def\leftouterjoin{\mathbin{\ojoin\mkern-5.8mu\bowtie}}
\def\rightouterjoin{\mathbin{\bowtie\mkern-5.8mu\ojoin}}
\def\fullouterjoin{\mathbin{\ojoin\mkern-5.8mu\bowtie\mkern-5.8mu\ojoin}}

\title{Relational Algebra Masterclass}
\date{}

\begin{document}

\maketitle

{\large {\bf Given:}}\\

Student(\underline{SID}, Sname, GPA)\\

Department(\underline{DName}, Chair, Building, Room)\\

Course(\underline{DName}, \underline{CID}, CName, Hours)\\

Enrolled(\underline{DName}, \underline{CID}, \underline{SID})\\

where
\begin{itemize}
\item DName in Course is a foreign key referencing Department,
\item DName,CID in Enrolled is a foreign key referencing Course,
\item SID in Enrolled is a foreign key referencing Student, and
\item primary keys are underlined.
\end{itemize}

and the relation states\\

\begin{multicols}{2}

Student\\
\begin{tabular}{|l|l|l|}\hline
\rowcolor{lightgray}  SID & Sname & GPA \\\hline
11 & Bush & 3.0 \\\hline
12 & Cruz & 3.2 \\\hline
13 & Clinton & 3.9 \\\hline
22 & Sanders& 3.0 \\\hline
33 & Trump & 3.8 \\\hline
\end{tabular}

\columnbreak

Enrolled\\
\begin{tabular}{|l|l|l|}\hline
\rowcolor{lightgray} DName & CID & SID \\\hline
CS & 101 & 11 \\\hline
Math & 101 & 11 \\\hline
CS & 101 & 12 \\\hline
CS & 101 & 22 \\\hline
Math & 103 & 33 \\\hline
EE & 102 & 33 \\\hline
CS & 102 & 22 \\\hline
\end{tabular}

\end{multicols}

\vspace{.2in}

\begin{multicols}{2}
Department\\
\begin{tabular}{|l|l|l|l|}\hline
\rowcolor{lightgray} DName & Chair & Building & Room \\\hline
CS & Rubio & Ajax & 100 \\\hline
Math & Carson & Acme & 300 \\\hline
EE & Kasich & Ajax & 200 \\\hline
Music & Costello & North & 100  \\\hline
\end{tabular}

\columnbreak

Course\\
\begin{tabular}{|l|l|l|l|}\hline
\rowcolor{lightgray} DName & CID & CName & Hours \\\hline
CS & 101 & Programming & 4 \\\hline
CS & 102 & Algorithms & 3 \\\hline
Math & 101 & Algebra & 3 \\\hline
Math & 103 & Calculus & 4 \\\hline
Music & 104 & Jazz & 3 \\\hline
EE & 102 & Circuits & 3 \\\hline
\end{tabular}

\end{multicols}

\newpage

{\bf Show how the following relational algebra expression gives the names of all students enrolled in two or more courses.}
\[
\pi_{SName}(\pi_{SID} (\sigma_{DName \ne D \text{ OR } CID \ne C} (\rho_{(D, C, SID)}(Enrolled) * Enrolled)) * Student)
\]

\begin{quote}
  Work from the inside out and play close attention to parentheses and operator-operand binding.
\end{quote}

Apply $\rho_{(D, C, SID)}(Enrolled)$, which creates a relation like $Enrolled$ but with $Dname$ and $CID$ renamed:\\

\begin{tabular}{|l|l|l|}\hline
\rowcolor{lightgray} D & C & SID \\\hline
CS & 101 & 11 \\\hline
Math & 101 & 11 \\\hline
CS & 101 & 12 \\\hline
CS & 101 & 22 \\\hline
Math & 103 & 33 \\\hline
EE & 102 & 33 \\\hline
CS & 102 & 22 \\\hline
\end{tabular}\\

Then apply $\rho_{(D, C, SID)}(Enrolled) * Enrolled$, which natural joins the relation created above with $Enrolled$:\\

\begin{tabular}{|l|l|l|l|l|}\hline
\rowcolor{lightgray} D & C & SID & DName & CID \\\hline
CS & 101 & 11 & CS & 101 \\\hline
CS & 101 & 11 & Math & 101 \\\hline
Math & 101 & 11 & CS & 101 \\\hline
Math & 101 & 11 & Math & 101\\\hline
CS & 101 & 12 & CS & 101 \\\hline
CS & 101 & 22 & CS & 101 \\\hline
CS & 101 & 22 & CS & 102 \\\hline
Math & 103 & 33 & Math & 103\\\hline
Math & 103 & 33 & EE & 102 \\\hline
EE & 102 & 33 & Math & 103 \\\hline
EE & 102 & 33 & EE & 102 \\\hline
CS & 102 & 22 & CS & 101 \\\hline
CS & 102 & 22 & CS & 102 \\\hline
\end{tabular}\\

Then apply $\sigma_{DName \ne D \text{ OR } CID \ne C} (\rho_{(D, C, SID)}(Enrolled) * Enrolled)$, which selects from the previous result only the rows for which $Dname \ne D$ or $CID \ne C$:

\begin{quote}
Tip: You can view selection as choosing tuples for inclusion, or choosing tuples for elimination by negating the $\theta$ condition. By DeMorgan's Law $\lnot\theta$ is $Dname = D \land CID = C$.
\end{quote}

\begin{tabular}{|l|l|l|l|l|}\hline
\rowcolor{lightgray} D & C & SID & DName & CID \\\hline
CS & 101 & 11 & Math & 101 \\\hline
Math & 101 & 11 & CS & 101 \\\hline
CS & 101 & 22 & CS & 102 \\\hline
Math & 103 & 33 & EE & 102 \\\hline
EE & 102 & 33 & Math & 103 \\\hline
CS & 102 & 22 & CS & 101 \\\hline
\end{tabular}\\

From that result we project the SID attribute by applying $\pi_{SID} (...)$, which gives us:\\

\begin{tabular}{|l|}\hline
\rowcolor{lightgray} SID  \\\hline
11 \\\hline
22 \\\hline
33 \\\hline
\end{tabular}\\

From there you can easily see that we have the $SID$s of all the students enrolled in two or more courses, which we natrual join with $Student$ so we can project the $SName$s for the final result.

%% \section{The lowest GPA of any student}

%% \[
%% \pi_{GPA}(Student) - \pi_{GPA}(\sigma_{GPA>G}(Student \times \rho_{(G)}(\pi_{GPA}(Student))))
%% \]


%% \section{Students who don't have he lowest GPA}

%% \[
%% \pi_{SName}(\sigma_{GPA>G}(\rho_{(G)}(\pi_{GPA}(Student)) \times Student))
%% \]

%% \begin{tabular}{|l|}\hline
%%   Cruz \\\hline
%%   Clinton \\\hline
%%   Trump \\\hline
%% \end{tabular}


%% \section{The SName of each student that has a GPA $>$ 3.0 and is enrolled in at least 1 course in the CS department}

%% \begin{itemize}
%% \item $\pi_{SName}(\sigma_{GPA > 3.0}(Student) * \sigma_{DName = 'CS'}(Enrolled))$
%% \item $\pi_{SName}(\sigma_{GPA > 3.0 \text{ AND } DName = 'CS'}(Student * Enrolled))$
%% \item $\pi_{SName}(\pi_{SName,SID}(\sigma_{GPA > 3.0}(Student)) * \pi_{SID}(\sigma_{DName = 'CS'}(Enrolled)))$
%% \end{itemize}

\end{document}
