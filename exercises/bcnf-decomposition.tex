\documentclass[10pt]{article}

\usepackage{verbatim, multicol, tabularx, graphicx, float, xcolor, colortbl}
\usepackage{amsmath,amsthm, amssymb, latexsym, listings, qtree}

\lstset{frame=tb,
  language=Java,
  aboveskip=1mm,
  belowskip=0mm,
  showstringspaces=false,
  columns=flexible,
  basicstyle={\ttfamily},
  numbers=none,
  frame=single,
  breaklines=true,
  breakatwhitespace=true
}

\textwidth = 6.5 in
\textheight = 9 in
\oddsidemargin = 0.0 in
\evensidemargin = 0.0 in
\topmargin = -0.25 in
\headheight = 0.0 in
\headsep = 0.0 in
\parskip = 0.0in
\parindent = 0.0in

\def\ojoin{\setbox0=\hbox{$\bowtie$}%
  \rule[-.02ex]{.25em}{.4pt}\llap{\rule[\ht0]{.25em}{.4pt}}}
\def\leftouterjoin{\mathbin{\ojoin\mkern-5.8mu\bowtie}}
\def\rightouterjoin{\mathbin{\bowtie\mkern-5.8mu\ojoin}}
\def\fullouterjoin{\mathbin{\ojoin\mkern-5.8mu\bowtie\mkern-5.8mu\ojoin}}

\title{BCNF Decomposition Exercise}
\date{}

\begin{document}

\maketitle

Given the universal relation schema:

\[
R(eid, pid, hours, ename, city, did, dname, mgrid, pname, ploc)
\]

the FDs:

\begin{minipage}{7cm}

\begin{eqnarray}
eid, pid & \rightarrow & hours\\
eid, pid & \rightarrow & dname\\
eid      & \rightarrow & ename, city, did, mgrid\\
did      & \rightarrow & dname, mgrid\\
pid      & \rightarrow & pname, ploc
\end{eqnarray}

\end{minipage}\\

and the data:\\

\begin{tabular}{|c|c|c|c|c|c|c|c|c|c|}\hline
\rowcolor{lightgray}eid & pid & hours & ename & city & did & dname & mgrid & pname & ploc \\\hline
101 & 1   & 20    & smith & atl & 10  & toy   & 103   & acme  & atl \\\hline
101 & 2   & 25    & smith & atl & 10  & toy   & 103   & ajax  & chi \\\hline
102 & 1   & 40    & jones & mac & 15  & shoe  & 105   & acme  & atl \\\hline
103 & 2   & 25    & brown & mar & 10  & toy   & 103   & ajax  & chi \\\hline
103 & 3   & 25    & brown & mar & 10  & toy   & 103   & aaa   & mia \\\hline
104 & 1   & 40    & green & mac & 15  & shoe  & 105   & acme  & atl \\\hline
105 & 2   & 40    & black & atl & 15  & shoe  & 105   & ajax  & chi \\\hline
\end{tabular}\\

\begin{itemize}
\item Is the set of functional dependencies above a minimal cover set?
\item What is the key of $R$?
\item Decompose the universal relation schema $R$ into BCNF relation schemas and show how the data above would be stored in states of the new set of relation schemas.
\end{itemize}
\end{document}
