\documentclass[answers,10pt,a4paper]{exam}

\usepackage{verbatim, multicol, tabularx, graphicx, float, xcolor, colortbl}
\usepackage{amsmath,amsthm, amssymb, latexsym, listings, qtree}

\lstset{frame=tb,
  language=Java,
  aboveskip=1mm,
  belowskip=0mm,
  showstringspaces=false,
  columns=flexible,
  basicstyle={\ttfamily},
  numbers=none,
  frame=single,
  breaklines=true,
  breakatwhitespace=true
}

%% \textwidth = 6.5 in
%% \textheight = 9 in
%% \oddsidemargin = 0.0 in
%% \evensidemargin = 0.0 in
%% \topmargin = -0.25 in
%% \headheight = 0.0 in
%% \headsep = 0.0 in
%% \parskip = 0.0in
%% \parindent = 0.0in

\def\ojoin{\setbox0=\hbox{$\bowtie$}%
  \rule[-.02ex]{.25em}{.4pt}\llap{\rule[\ht0]{.25em}{.4pt}}}
\def\leftouterjoin{\mathbin{\ojoin\mkern-5.8mu\bowtie}}
\def\rightouterjoin{\mathbin{\bowtie\mkern-5.8mu\ojoin}}
\def\fullouterjoin{\mathbin{\ojoin\mkern-5.8mu\bowtie\mkern-5.8mu\ojoin}}

\def\a{& $\blacksquare\blacksquare\blacksquare$ & [ B ] & [ C ] & [ D ] \\}
\def\b{& [ A ] & $\blacksquare\blacksquare\blacksquare$ & [ C ] & [ D ] \\}
\def\c{& [ A ] & [ B ] & $\blacksquare\blacksquare\blacksquare$ & [ D ] \\}
\def\d{& [ A ] & [ B ] & [ C ] & $\blacksquare\blacksquare\blacksquare$ \\}

\title{CS 4400 Exam 2}
\date{Practice}
\setcounter{page}{0}
\begin{document}

\maketitle
\thispagestyle{empty}
\firstpageheader{}               {\tiny Copyright \textcopyright\ 2016 All rights reserved. Duplication and/or usage for purposes of any kind without permission is strictly forbidden.}
                {}


\runningheader{}
{\small Name: \underline{\hspace{2.8in}} GTAccount: \underline{\hspace{1.4in}} Section: \underline{\hspace{.5in}}}
{}

%% \footer{Page \thepage\ of \numpages}
%%               {}
%%               {Points available: \pointsonpage{\thepage} -
%%                points lost: \makebox[.5in]{\hrulefill} =
%%                points earned:  \makebox[.5in]{\hrulefill}.
%%               Graded by: \makebox[.5in]{\hrulefill}}


\ifprintanswers
\begin{center}
{\LARGE ANSWER KEY}
\end{center}
\vspace{.25in}
\else
\vspace{0.1in}
\hbox to \textwidth{Name: \enspace\hrulefill}
\vspace{0.2in}
\hbox to \textwidth{GT account (gtg, gth, msmith3, etc): \underline{\hspace{2in}} Section (e.g., B1): \underline{\hspace{.75in}}}
\vspace{0.2in}
\hbox to \textwidth{Signature: \enspace\hrulefill}

\vfill

\begin{itemize}
\item Failure to properly fill in the information on this page will result in a deduction of up to 4 points from your exam score.
\item Signing signifies that you agree to comply with the {\bf Academic Honor Code of Georgia Tech}.
\item Calculators and cell phones are NOT allowed.
\end{itemize}

\fi


% Points Table
%\begin{center}
%\addpoints
%\gradetable[v][pages]
%\end{center}

Completely fill in the box corresponding to your answer choice for each question.

\ifprintanswers
\begin{tabular}{lcccc}\\
  1. \d
  2. \a
  3. \d
  4. \d
  5. \d
  6. \b
  7. \a
  8. \b
  9. \b
  10. \a
  11. \b
  12. \b
  13. \b
  14. \a
  15. \d
  16. \b
  17. \a
  18. \b
  19. \b
  20. \a
  21. \c
  22. \b
  23. \b
  24. \b
  25. \c
\end{tabular}
\else
\begin{tabular}{lcccc}\\
  1. & [ A ] & [ B ] & [ C ] & [ D ] \\
  2. & [ A ] & [ B ] & [ C ] & [ D ] \\
  3. & [ A ] & [ B ] & [ C ] & [ D ] \\
  4. & [ A ] & [ B ] & [ C ] & [ D ] \\
  5. & [ A ] & [ B ] & [ C ] & [ D ] \\
  6. & [ A ] & [ B ] & [ C ] & [ D ] \\
  7. & [ A ] & [ B ] & [ C ] & [ D ] \\
  8. & [ A ] & [ B ] & [ C ] & [ D ] \\
  9. & [ A ] & [ B ] & [ C ] & [ D ] \\
  10. & [ A ] & [ B ] & [ C ] & [ D ] \\
  11. & [ A ] & [ B ] & [ C ] & [ D ] \\
  12. & [ A ] & [ B ] & [ C ] & [ D ] \\
  13. & [ A ] & [ B ] & [ C ] & [ D ] \\
  14. & [ A ] & [ B ] & [ C ] & [ D ] \\
  15. & [ A ] & [ B ] & [ C ] & [ D ] \\
  16. & [ A ] & [ B ] & [ C ] & [ D ] \\
  17. & [ A ] & [ B ] & [ C ] & [ D ] \\
  18. & [ A ] & [ B ] & [ C ] & [ D ] \\
  19. & [ A ] & [ B ] & [ C ] & [ D ] \\
  20. & [ A ] & [ B ] & [ C ] & [ D ] \\
  21. & [ A ] & [ B ] & [ C ] & [ D ] \\
  22. & [ A ] & [ B ] & [ C ] & [ D ] \\
  23. & [ A ] & [ B ] & [ C ] & [ D ] \\
  24. & [ A ] & [ B ] & [ C ] & [ D ] \\
  25. & [ A ] & [ B ] & [ C ] & [ D ] \\
\end{tabular}

\fi

\vspace{.5in}

Number missed: \makebox[.5in]{\hrulefill} Final Score: \makebox[.5in]{\hrulefill}

\newpage

%\normalsize

\pointsinmargin
\bracketedpoints

\marginpointname{}
%%%%%%%%%%%%%%%%%%%%%%%%%%%%%%%%%%%%%%%%%%%%%%%%%%%%%%%%%%%%%%%%%%%%%%%%%%%%

\begin{figure}[H]

\section*{Pubs Database Schema}

$author(\underline{author\_id}, first\_name, last\_name)$\\

$author\_pub(\underline{author\_id}, \underline{pub\_id}, author\_position)$\\

$book(\underline{book\_id}, book\_title, month, year, editor)$\\

$pub(\underline{pub\_id}, title, book\_id)$

\begin{itemize}
\item $author\_id$ in $author\_pub$ is a foreign key referencing $author$
\item $pub\_id$ in $author\_pub$ is a foreign key referencing $pub$
\item $book\_id$ in $pub$ is a foreign key referencing $book$
\item $editor$ in $book$ is a foreign key referencing $author(author\_id)$
\item Primary keys are underlined
\end{itemize}

\section*{Pubs Database State}

\begin{multicols}{2}

$r(author)$\\
\begin{tabular}{|l|l|l|}\hline
\rowcolor{lightgray} author\_id & first\_name & last\_name \\\hline
1 & John       & McCarthy  \\\hline
2 & Dennis     & Ritchie   \\\hline
3 & Ken        & Thompson  \\\hline
4 & Claude     & Shannon   \\\hline
5 & Alan       & Turing    \\\hline
6 & Alonzo     & Church    \\\hline
7 & Perry      & White     \\\hline
8 & Moshe      & Vardi     \\\hline
9 & Roy        & Batty     \\\hline
\end{tabular}

\columnbreak

$r(author\_pub)$\\
\begin{tabular}{|l|l|l|}\hline
\rowcolor{lightgray} author\_id & pub\_id & author\_position \\\hline
1 &      1 &      1 \\\hline
2 &      2 &      1 \\\hline
3 &      2 &      2 \\\hline
4 &      3 &      1 \\\hline
5 &      4 &      1 \\\hline
5 &      5 &      1 \\\hline
6 &      6 &      1 \\\hline
\end{tabular}
\end{multicols}

\begin{multicols}{2}

$r(book)$\\
\begin{tabular}{|l|l|l|l|l|}\hline
\rowcolor{lightgray} book\_id & book\_title & month    & year & editor \\\hline
       1 & CACM       & April    & 1960 &      8 \\\hline
       2 & CACM       & July     & 1974 &      8 \\\hline
       3 & BST        & July     & 1948 &      2 \\\hline
       4 & LMS        & November & 1936 &      7 \\\hline
       5 & Mind       & October  & 1950 &   NULL \\\hline
       6 & AMS        & Month    & 1941 &   NULL \\\hline
       7 & AAAI       & July     & 2012 &      9 \\\hline
       8 & NIPS       & July     & 2012 &      9 \\\hline
\end{tabular}

\columnbreak

$r(pub)$\\
\begin{tabular}{|l|l|l|}\hline
\rowcolor{lightgray} pub\_id & title           & book\_id \\\hline
     1 & LISP            &       1 \\\hline
     2 & Unix            &       2 \\\hline
     3 & Info Theory     &       3 \\\hline
     4 & Turing Machines &       4 \\\hline
     5 & Turing Test     &       5 \\\hline
     6 & Lambda Calculus &       6 \\\hline
\end{tabular}

\end{multicols}



\caption{Relational Database Schema}
\label{fig:db-schema}
\end{figure}

\newpage

Scratch page

\newpage

\begin{questions}

\question[4] Which of the following statements is true with regard to the relational data model?

\begin{choices}
\choice A domain for an attribute is a set of atomic values.
\choice Several attributes in one relation schema may have the same domain.
\choice A tuple in a relation consists of one value from each attribute domain of that relation.
\correctchoice All of the above
\end{choices}

\question[4] Which of the following is the mathematical definition of a relation, $r(R)$, of degree $n$?

\begin{choices}
\correctchoice $r(R) \subseteq dom(A_1) \times dom(A_2) \times .... \times dom(A_n)$
\choice $r(R) \subseteq dom(A_1) \cap dom(A_2) \cap .... \cap dom(A_n)$
\choice $r(R) \subseteq dom(A_1) \cup dom(A_2) \cup .... \cup dom(A_n)$
\choice none of the above
\end{choices}

\question[4] Which of the following are properties of the relational model?

\begin{choices}
\choice Attribute values in tuples are indivisible.
\choice Facts not asserted explicitly are assumed to be false.
\choice Relations are sets.
\correctchoice All of the above.
\end{choices}

\question[4] Which of the following is true about a minimal superkey?

\begin{choices}
\choice There can be only one.
\choice The default superkey is always a minimal superkey.
\choice Every minimal superkey is a primary key.
\correctchoice Every superkey contains a minimal superkey as a subset.
\end{choices}

\question[4] In a relation schema with 3 attributes, each of which is a candidate key, how many superkeys are there?

\begin{choices}
\choice 1
\choice 3
\choice 6
\correctchoice 7
\end{choices}

\question[4] In a relation schema with 3 attributes, each of which is a candidate key, how many choices are there for the primary key?

\begin{choices}
\choice 1
\correctchoice 3
\choice 6
\choice 7
\end{choices}

\question[4] May a tuple in a relation have a NULL value for a foreign key attribute?

\begin{choices}
\correctchoice Yes
\choice No
\end{choices}

\question[4] May a tuple in a relation have a NULL value for a primary key attribute?

\begin{choices}
\choice Yes
\correctchoice No
\end{choices}

\question[4] Which kind of constraint cannot be specied in the relational model?

\begin{choices}
\choice referential integrity constraints
\correctchoice semantic constraints, a.k.a., business rules
\choice entity integrity constraints
\end{choices}

\question[4] Meow!

\begin{choices}
\correctchoice True
\end{choices}


\newpage

Refer to database schema in Figure 1 for the remaining questions.

\question[4] What is the degree of the $author$ relation?

\begin{choices}
\choice 2
\correctchoice 3
\choice 9
\end{choices}

\question[4] The $author\_pub$ relation has how many superkeys?

\begin{choices}
\choice 1
\correctchoice 2
\choice 3
\end{choices}

\question[4] Can the tuple {\tt <6, 'Teen', 'Candles'>} be inserted into the $author$ relation without causing an integrity violation?

\begin{choices}
\choice Yes
\correctchoice No
\end{choices}

\question[4] Can the tuple {\tt <10, NULL, 'Pointers'>} be inserted into the $author$ relation without causing an integrity violation?

\begin{choices}
\correctchoice Yes
\choice No
\end{choices}

\question[4] The deletion of the second tuple in the $author$ relation ({\tt <2, 'Dennis', 'Ritchie'>}) causes an integrity violation for which relations?

\begin{choices}
\choice $author\_pub$
\choice $book$
\choice $pub$
\correctchoice A and B above.
\end{choices}


\question[4] If cascading deletes is in effect for all relations and the tuple {\tt <2, 'Dennis', 'Ritchie'>} is deleted, how many other tuples will be deleted from the database?

\begin{choices}
\choice 0
\correctchoice 2
\choice 3
\end{choices}

\question[4] How many tuples will be returned by the following relational algebra query?

\[
\pi_{book\_title}(book)
\]

\begin{choices}
\correctchoice 7
\choice 5
\choice 2
\choice 1
\end{choices}

\newpage

\question[4] What question does the following expression answer?

\[
|\pi_{author\_id}(author)  -  \pi_{editor}(book)|
\]

\begin{choices}
\choice How many authors are book editors.
\correctchoice How many authors are not book editors.
\choice What are the names of the authors who are book editors.
\choice What are the names of the authors who are not book editors.
\end{choices}

\question[4] Which of the following relational algebra expressions returns the names of all authors who are book editors?

\begin{choices}
\choice $\pi_{first\_name, last\_name}((\pi_{author\_id}(author)  -  \pi_{editor}(book)) * author)$
\correctchoice $\pi_{first\_name, last\_name}(author \bowtie_{author\_id = editor} book)$
\choice $\pi_{first\_name, last\_name}(author * author\_pub)$
\end{choices}

\question[4] Which of the following relational algebra expressions returns the names of all authors who are {\bf not} book editors?

\begin{choices}
\correctchoice $\pi_{first\_name, last\_name}((\pi_{author\_id}(author)  -  \pi_{editor}(book)) * author)$
\choice $\pi_{first\_name, last\_name}(author \bowtie_{author\_id = editor} book)$
\choice $\pi_{first\_name, last\_name}(author * author\_pub)$
\end{choices}


\question[4] Which of the following relational algebra expressions returns the names of all authors who have at least one publication in the database?

\begin{choices}
\choice $\pi_{first\_name, last\_name}((\pi_{author\_id}(author)  -  \pi_{editor}(book)) * author)$
\choice $\pi_{first\_name, last\_name}(author \bowtie_{author\_id = editor} book)$
\correctchoice $\pi_{first\_name, last\_name}(author * author\_pub)$
\end{choices}


\question[4] Which of the following relational algebra expressions returns books that were published before 1960 or after 2000?

\begin{choices}
\choice $\sigma_{year < 1960}(book) \land \sigma_{year > 2000}(book)$
\correctchoice $\sigma_{year < 1960}(book) \cup \sigma_{year > 2000}(book)$
\choice $\sigma_{year < 1960 \land year > 2000}(book)$
\end{choices}

\question[4] How many tuples are returned by the following relational algebra expression?

\[
author \leftouterjoin_{author\_id = editor} book
\]

\begin{choices}
\choice 8
\correctchoice 11
\choice 13
\end{choices}

\question[4] What question does the following relational algebra expression answer?

\[
author * (author\_pub * (\sigma_{month = 'July'}(book) * pub))
\]

\begin{choices}
\choice Which authors were born in July?
\correctchoice Which authors authored a pub that was published in July?
\choice Which authors edited books that were published in July?
\end{choices}

\question[4] How many tuples does the previous relational algebra expression return?

\begin{choices}
\choice 1
\choice 2
\correctchoice 3
\choice 4
\end{choices}


%%%%%%%%%%%%%%%%%%%%%%%%%%%%%%%%%%%%%%%%%%%%%%%%%%%%%%%%%%%%%%%%%%%%%%%%%%%%%%%%%
% Relational calculus questions. Uncomment in future semester.

%% \question[4] Which relational tuple calculus query produces the same result as the following relational algebra query?

%% \[
%% \pi_{CID,CName}(\sigma_{Hours > 3 \text{ AND } DName = 'CS'}(Course))
%% \]

%% \begin{choices}
%%   \choice $\{t.CID, t.CName | Course(t) \text{ AND } t.Hours > 3 \text{ AND } t.DName = 'CS'\}$
%%   \choice $\{t.CID, t.CName | Course(t)\}$
%%   \choice $\{t.CID, t.CName | Course(t) \text{ AND } (\exists s)(Course(s) \text{ AND } t.Hours > 3 \text{ AND } t.CID = s.CID)\}$
%%   \choice None of the above
%% \end{choices}

%% \question[4] What is the result of executing the following relational tuple calculus query?

%% \[
%% \{t.Chair | Department(t) \text{ AND } (\exists s)(Enrolled(s) \text{ AND } t.DName = s.DName)\}
%% \]

%% \begin{choices}
%% \choice
%% \begin{tabular}{|l|}\hline
%% Costello \\\hline
%% \end{tabular}

%% \correctchoice
%% \begin{tabular}{|l|}\hline
%%   Rubio \\\hline
%%   Carson \\\hline
%%   Kasich \\\hline
%% \end{tabular}

%% \choice
%% \begin{tabular}{|l|}\hline
%%   Rubio \\\hline
%%   Rubio \\\hline
%%   Kasich \\\hline
%%   Rubio \\\hline
%%   Carson \\\hline
%% \end{tabular}

%% \choice None of the above
%% \end{choices}

%% \question[4] How many tuples appear in the result of the following relational tuple calculus query?

%% \[
%% \{t | Department(t) \text{ AND } (\exists s)(Student(s))\}
%% \]

%% \begin{choices}
%% \choice 28
%% \choice 7
%% \correctchoice 4
%% \choice 0
%% \end{choices}

\end{questions}

\end{document}
